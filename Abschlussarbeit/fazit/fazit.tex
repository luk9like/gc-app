%
%%%%%%%%%%%%%%%%%%%%%%%%%%%%%%%%%%%%%%%%%%%%%%%%%%%%%%%%%%
%
%  Z U S A M M E N F A S S U N G
%
%%%%%%%%%%%%%%%%%%%%%%%%%%%%%%%%%%%%%%%%%%%%%%%%%%%%%%%%%%
\chapter{Schluss}
\label{cha:fazit}
%
Unter funktionalen Anforderungen versteht man konkrekte Funktionalitäten, welche die Anwendung leisten soll. Die Anwendung lässt sich in mehrere Module aufteilen, die dem User zu Verfügung stehen. Im Folgenden werden die funktionalen Anforderungen erläutert.
%
\section{Zusammenfassung}
%
Unter funktionalen Anforderungen versteht man konkrekte Funktionalitäten, welche die Anwendung leisten soll. Die Anwendung lässt sich in mehrere Module aufteilen, die dem User zu Verfügung stehen. Im Folgenden werden die funktionalen Anforderungen erläutert.
%
\paragraph{Angular}
Der Einstieg in das Framework war etwas mühsam. Angular bietet einen sehr großen Funktionsumfang, was den Einstieg für Neulinge etwas zeitintensiv gestaltet. Durch die klare Strukturierung lässt sich damit aber sehr gut arbeiten. Besonders die Angular CLI war sehr nützlich beim Entwickeln der Anwendung. Bei Problemen hat man recht schnell eine Lösung in der Community gefunden. Toll war auch die hohe Kompatibilität, beispielsweise mit Webpack. Gerade für größere Single Page Anwendungen, die im Team entwickelt werden, ist Angular sehr gut. Ein Nachteil war allerdings die TypeScript Programmiersprache. Das ist anfangs sehr gewöhnungsbedürftig und ist hier und da auch unpraktisch, wenn man JavaScript gewohnt ist. Funktionen aus der JavaScript-Welt müssen erst umgeschrieben werden, was natürlich Zeit kostet. Besonders bei der Verwendung von Three.js Beispielen war das etwas mühsam. Alles in allem schneidet das Framework bei der Umsetzung des Konfigurators gut ab.
\paragraph{Three.js}
Wie schon erwähnt, war das Zusammenspiel von Angular und Three.js nicht immer ganz optimal. Manche Zusatzmodule von Three.js konnten in Angular nicht direkt importiert werden. Dazu wurde eine zusätzliche Bibliothek benötigt. Für einen schnellen Einstieg mit Three.js sollte ein Grundverständnis von 3D-Modellierung vorhanden sein. Durch die objektorientierte Umsetzung ist die 3D-Bibliothek allerdings sehr einfach zu verwenden. 3D Modelle aus Rendering-Programmen können schnell in eine Szene geladen werden. Der Funktionsumfang ist auch sehr groß. Zwar gibt es hier und da noch ein paar Bugs, jedoch ist die Community sehr aktiv und man bekommt schnell eine Lösung. Three.js wurde im Rahmen des Projektes nicht gänzlich ausgeschöpft. Zusammenfassend lässt sich sagen, das es eine einfach zu bedienende Bibliothek ist, die trotzdem einen großen Funktionsumfang bietet.
\paragraph{Angular}
Der Einstieg in das Framework war etwas mühsam. Angular bietet einen sehr großen Funktionsumfang, was den Einstieg für Neulinge etwas zeitintensiv gestaltet. Durch die klare Strukturierung lässt sich damit aber sehr gut arbeiten. Besonders die Angular CLI war sehr nützlich beim Entwickeln der Anwendung. Bei Problemen hat man recht schnell eine Lösung in der Community gefunden. Toll war auch die hohe Kompatibilität, beispielsweise mit Webpack. Gerade für größere Single Page Anwendungen, die im Team entwickelt werden, ist Angular sehr gut. Ein Nachteil war allerdings die TypeScript Programmiersprache. Das ist anfangs sehr gewöhnungsbedürftig und ist hier und da auch unpraktisch, wenn man JavaScript gewohnt ist. Funktionen aus der JavaScript-Welt müssen erst umgeschrieben werden, was natürlich Zeit kostet. Besonders bei der Verwendung von Three.js Beispielen war das etwas mühsam. Alles in allem schneidet das Framework bei der Umsetzung des Konfigurators gut ab.
\paragraph{Three.js}
Wie schon erwähnt, war das Zusammenspiel von Angular und Three.js nicht immer ganz optimal. Manche Zusatzmodule von Three.js konnten in Angular nicht direkt importiert werden. Dazu wurde eine zusätzliche Bibliothek benötigt. Für einen schnellen Einstieg mit Three.js sollte ein Grundverständnis von 3D-Modellierung vorhanden sein. Durch die objektorientierte Umsetzung ist die 3D-Bibliothek allerdings sehr einfach zu verwenden. 3D Modelle aus Rendering-Programmen können schnell in eine Szene geladen werden. Der Funktionsumfang ist auch sehr groß. Zwar gibt es hier und da noch ein paar Bugs, jedoch ist die Community sehr aktiv und man bekommt schnell eine Lösung. Three.js wurde im Rahmen des Projektes nicht gänzlich ausgeschöpft. Zusammenfassend lässt sich sagen, das es eine einfach zu bedienende Bibliothek ist, die trotzdem einen großen Funktionsumfang bietet.
\section{Ausblick}
%
Unter funktionalen Anforderungen versteht man konkrekte Funktionalitäten, welche die Anwendung leisten soll. Die Anwendung lässt sich in mehrere Module aufteilen, die dem User zu Verfügung stehen. Im Folgenden werden die funktionalen Anforderungen erläutert.
%
\paragraph{Angular}
Der Einstieg in das Framework war etwas mühsam. Angular bietet einen sehr großen Funktionsumfang, was den Einstieg für Neulinge etwas zeitintensiv gestaltet. Durch die klare Strukturierung lässt sich damit aber sehr gut arbeiten. Besonders die Angular CLI war sehr nützlich beim Entwickeln der Anwendung. Bei Problemen hat man recht schnell eine Lösung in der Community gefunden. Toll war auch die hohe Kompatibilität, beispielsweise mit Webpack. Gerade für größere Single Page Anwendungen, die im Team entwickelt werden, ist Angular sehr gut. Ein Nachteil war allerdings die TypeScript Programmiersprache. Das ist anfangs sehr gewöhnungsbedürftig und ist hier und da auch unpraktisch, wenn man JavaScript gewohnt ist. Funktionen aus der JavaScript-Welt müssen erst umgeschrieben werden, was natürlich Zeit kostet. Besonders bei der Verwendung von Three.js Beispielen war das etwas mühsam. Alles in allem schneidet das Framework bei der Umsetzung des Konfigurators gut ab.
\paragraph{Three.js}
Wie schon erwähnt, war das Zusammenspiel von Angular und Three.js nicht immer ganz optimal. Manche Zusatzmodule von Three.js konnten in Angular nicht direkt importiert werden. Dazu wurde eine zusätzliche Bibliothek benötigt. Für einen schnellen Einstieg mit Three.js sollte ein Grundverständnis von 3D-Modellierung vorhanden sein. Durch die objektorientierte Umsetzung ist die 3D-Bibliothek allerdings sehr einfach zu verwenden. 3D Modelle aus Rendering-Programmen können schnell in eine Szene geladen werden. Der Funktionsumfang ist auch sehr groß. Zwar gibt es hier und da noch ein paar Bugs, jedoch ist die Community sehr aktiv und man bekommt schnell eine Lösung. Three.js wurde im Rahmen des Projektes nicht gänzlich ausgeschöpft. Zusammenfassend lässt sich sagen, das es eine einfach zu bedienende Bibliothek ist, die trotzdem einen großen Funktionsumfang bietet.
\paragraph{Angular}
Der Einstieg in das Framework war etwas mühsam. Angular bietet einen sehr großen Funktionsumfang, was den Einstieg für Neulinge etwas zeitintensiv gestaltet. Durch die klare Strukturierung lässt sich damit aber sehr gut arbeiten. Besonders die Angular CLI war sehr nützlich beim Entwickeln der Anwendung. Bei Problemen hat man recht schnell eine Lösung in der Community gefunden. Toll war auch die hohe Kompatibilität, beispielsweise mit Webpack. Gerade für größere Single Page Anwendungen, die im Team entwickelt werden, ist Angular sehr gut. Ein Nachteil war allerdings die TypeScript Programmiersprache. Das ist anfangs sehr gewöhnungsbedürftig und ist hier und da auch unpraktisch, wenn man JavaScript gewohnt ist. Funktionen aus der JavaScript-Welt müssen erst umgeschrieben werden, was natürlich Zeit kostet. Besonders bei der Verwendung von Three.js Beispielen war das etwas mühsam. Alles in allem schneidet das Framework bei der Umsetzung des Konfigurators gut ab.
\paragraph{Three.js}
Wie schon erwähnt, war das Zusammenspiel von Angular und Three.js nicht immer ganz optimal. Manche Zusatzmodule von Three.js konnten in Angular nicht direkt importiert werden. Dazu wurde eine zusätzliche Bibliothek benötigt. Für einen schnellen Einstieg mit Three.js sollte ein Grundverständnis von 3D-Modellierung vorhanden sein. Durch die objektorientierte Umsetzung ist die 3D-Bibliothek allerdings sehr einfach zu verwenden. 3D Modelle aus Rendering-Programmen können schnell in eine Szene geladen werden. Der Funktionsumfang ist auch sehr groß. Zwar gibt es hier und da noch ein paar Bugs, jedoch ist die Community sehr aktiv und man bekommt schnell eine Lösung. Three.js wurde im Rahmen des Projektes nicht gänzlich ausgeschöpft. Zusammenfassend lässt sich sagen, das es eine einfach zu bedienende Bibliothek ist, die trotzdem einen großen Funktionsumfang bietet.
\paragraph{Angular}
Der Einstieg in das Framework war etwas mühsam. Angular bietet einen sehr großen Funktionsumfang, was den Einstieg für Neulinge etwas zeitintensiv gestaltet. Durch die klare Strukturierung lässt sich damit aber sehr gut arbeiten. Besonders die Angular CLI war sehr nützlich beim Entwickeln der Anwendung. Bei Problemen hat man recht schnell eine Lösung in der Community gefunden. Toll war auch die hohe Kompatibilität, beispielsweise mit Webpack. Gerade für größere Single Page Anwendungen, die im Team entwickelt werden, ist Angular sehr gut. Ein Nachteil war allerdings die TypeScript Programmiersprache. Das ist anfangs sehr gewöhnungsbedürftig und ist hier und da auch unpraktisch, wenn man JavaScript gewohnt ist. Funktionen aus der JavaScript-Welt müssen erst umgeschrieben werden, was natürlich Zeit kostet. Besonders bei der Verwendung von Three.js Beispielen war das etwas mühsam. Alles in allem schneidet das Framework bei der Umsetzung des Konfigurators gut ab.
