%
%%%%%%%%%%%%%%%%%%%%%%%%%%%%%%%%%%%%%%%%%%%%%%%%%%%%%%%%%%
%
%  U M S E T Z U N G   &   I M P L E M E N T I E R U N G
%
%%%%%%%%%%%%%%%%%%%%%%%%%%%%%%%%%%%%%%%%%%%%%%%%%%%%%%%%%%
\chapter{Umsetzung \& Imlementierung}
\label{cha:umsetzung}
%
Unter funktionalen Anforderungen versteht man konkrekte Funktionalitäten, welche die Anwendung leisten soll. Die Anwendung lässt sich in mehrere Module aufteilen, die dem User zu Verfügung stehen. Im Folgenden werden die funktionalen Anforderungen erläutert.
\section{Design Rendering}
\label{sec:umsetzung}
%
Diese Vorlage verwendet die \textit{memoir}-LaTeX-Klasse von Peter Wilson und erzeugt somit ein Buchdokument, das auch zweiseitig gedruckt werden sollte, da zwischen den Kapiteln Leerseiten erzeugt werden. Einstiegspunkt für die Bachelorarbeit ist die Datei \texttt{abschlussarbeit.tex}. Tragen Sie zunächst den Titel der Arbeit und alle Namen ein. Bei hochschulinternen Bachlorarbeiten fallen die Eintragungen \textit{durchgeführt am} und \textit{Betreuer in der Firma} natürlich weg. 
\paragraph{3D Zylinder für Design rendering}
%https://threejs.org/docs/#api/en/geometries/CylinderBufferGeometry\\
%https://threejs.org/docs/#api/en/core/BufferGeometry\\
%https://codepen.io/JasonStoltz/pen/NaBOGm (transparente Designs) \\
%https://www.flyeralarm.com/sheets/de/becher_kstoff_200ml.pdf\\
%https://stackoverflow.com/questions/55283187/threejs-texture-fit-uv-map\\