%
%%%%%%%%%%%%%%%%%%%%%%%%%%%%%%%%%%%%%%%%%%%%%%%%%%%%%%%%%%
%
%  U M S E T Z U N G   &   I M P L E M E N T I E R U N G
%
%%%%%%%%%%%%%%%%%%%%%%%%%%%%%%%%%%%%%%%%%%%%%%%%%%%%%%%%%%
\chapter{Umsetzung \& Imlementierung}
\label{cha:umsetzung}
%
Unter funktionalen Anforderungen versteht man konkrekte Funktionalitäten, welche die Anwendung leisten soll. Die Anwendung lässt sich in mehrere Module aufteilen, die dem User zu Verfügung stehen. Im Folgenden werden die funktionalen Anforderungen erläutert.
\section{Entwicklungsumgebung}
\label{sec:umsetzung}
%
Diese Vorlage verwendet die \textit{memoir}-LaTeX-Klasse von Peter Wilson und erzeugt somit ein Buchdokument, das auch zweiseitig gedruckt werden sollte, da zwischen den Kapiteln Leerseiten erzeugt werden. Einstiegspunkt für die Bachelorarbeit ist die Datei \texttt{abschlussarbeit.tex}. Tragen Sie zunächst den Titel der Arbeit und alle Namen ein. Bei hochschulinternen Bachlorarbeiten fallen die Eintragungen \textit{durchgeführt am} und \textit{Betreuer in der Firma} natürlich weg.
%
\section{3D Szene mit Model}
\label{sec:umsetzung}
%
Diese Vorlage verwendet die \textit{memoir}-LaTeX-Klasse von Peter Wilson und erzeugt somit ein Buchdokument, das auch zweiseitig gedruckt werden sollte, da zwischen den Kapiteln Leerseiten erzeugt werden. Einstiegspunkt für die Bachelorarbeit ist die Datei \texttt{abschlussarbeit.tex}. Tragen Sie zunächst den Titel der Arbeit und alle Namen ein. Bei hochschulinternen Bachlorarbeiten fallen die Eintragungen \textit{durchgeführt am} und \textit{Betreuer in der Firma} natürlich weg.
\paragraph{3D-Szenen-Service}

\section{Module des Konfigurators}
\label{sec:umsetzung}
%
Im folgenden Kapitel wird betrachtet aus welchen Module die Angular Anwendung aufgebaut wird. Anschließend werden die Module angelegt. Es geht hier nur um das Anlegen der Module, noch nicht um die Funktionalität der einzelnen Module. Die Abbildung zeigt eine Übersicht aller Module, welche in der Webanwendung implementiert sind.
Die einzelnen Bausteine der Angular-Anwendung werden im Rahmen dieses Bachlorprojektes immer Über die Angular CLI generiert, da das am einfachsten funktioniert und zeitsparend ist. Ein Modul kann mit folgendem Command Line Befehl generiert werden:\\

\texttt{ng m module-name}\\

Es gibt noch ein paar Zusatzoptionen, die als Parameter angegben werden können. Diese sind in der offizielen Dokumentation von Angular aufgeführt.
 
Diese Vorlage verwendet die \textit{memoir}-LaTeX-Klasse von Peter Wilson und erzeugt somit ein Buchdokument, das auch zweiseitig gedruckt werden sollte, da zwischen den Kapiteln Leerseiten erzeugt werden. Einstiegspunkt für die Bachelorarbeit ist die Datei \texttt{abschlussarbeit.tex}. Tragen Sie zunächst den Titel der Arbeit und alle Namen ein. Bei hochschulinternen Bachlorarbeiten fallen die Eintragungen \textit{durchgeführt am} und \textit{Betreuer in der Firma} natürlich weg.

\section{Komponenten des Konfigurators}
\label{sec:umsetzung}
%
Diese Vorlage verwendet die \textit{memoir}-LaTeX-Klasse von Peter Wilson und erzeugt somit ein Buchdokument, das auch zweiseitig gedruckt werden sollte, da zwischen den Kapiteln Leerseiten erzeugt werden. Einstiegspunkt für die Bachelorarbeit ist die Datei \texttt{abschlussarbeit.tex}. Tragen Sie zunächst den Titel der Arbeit und alle Namen ein. Bei hochschulinternen Bachlorarbeiten fallen die Eintragungen \textit{durchgeführt am} und \textit{Betreuer in der Firma} natürlich weg.

\section{Design Rendering}
\label{sec:umsetzung}
%
Diese Vorlage verwendet die \textit{memoir}-LaTeX-Klasse von Peter Wilson und erzeugt somit ein Buchdokument, das auch zweiseitig gedruckt werden sollte, da zwischen den Kapiteln Leerseiten erzeugt werden. Einstiegspunkt für die Bachelorarbeit ist die Datei \texttt{abschlussarbeit.tex}. Tragen Sie zunächst den Titel der Arbeit und alle Namen ein. Bei hochschulinternen Bachlorarbeiten fallen die Eintragungen \textit{durchgeführt am} und \textit{Betreuer in der Firma} natürlich weg. 
\paragraph{3D Zylinder für Design rendering}
%https://threejs.org/docs/#api/en/geometries/CylinderBufferGeometry\\
%https://threejs.org/docs/#api/en/core/BufferGeometry\\
%https://codepen.io/JasonStoltz/pen/NaBOGm (transparente Designs) \\
%https://www.flyeralarm.com/sheets/de/becher_kstoff_200ml.pdf\\
%https://stackoverflow.com/questions/55283187/threejs-texture-fit-uv-map\\

\section{Mobile Navigation}
\label{sec:umsetzung}
%
Diese Vorlage verwendet die \textit{memoir}-LaTeX-Klasse von Peter Wilson und erzeugt somit ein Buchdokument, das auch zweiseitig gedruckt werden sollte, da zwischen den Kapiteln Leerseiten erzeugt werden. Einstiegspunkt für die Bachelorarbeit ist die Datei \texttt{abschlussarbeit.tex}. Tragen Sie zunächst den Titel der Arbeit und alle Namen ein. Bei hochschulinternen Bachlorarbeiten fallen die Eintragungen \textit{durchgeführt am} und \textit{Betreuer in der Firma} natürlich weg.

\section{Upload-Vorgang}
\label{sec:umsetzung}
%
Diese Vorlage verwendet die \textit{memoir}-LaTeX-Klasse von Peter Wilson und erzeugt somit ein Buchdokument, das auch zweiseitig gedruckt werden sollte, da zwischen den Kapiteln Leerseiten erzeugt werden. Einstiegspunkt für die Bachelorarbeit ist die Datei \texttt{abschlussarbeit.tex}. Tragen Sie zunächst den Titel der Arbeit und alle Namen ein. Bei hochschulinternen Bachlorarbeiten fallen die Eintragungen \textit{durchgeführt am} und \textit{Betreuer in der Firma} natürlich weg.

\section{Daten-Service}
\label{sec:umsetzung}
%
Diese Vorlage verwendet die \textit{memoir}-LaTeX-Klasse von Peter Wilson und erzeugt somit ein Buchdokument, das auch zweiseitig gedruckt werden sollte, da zwischen den Kapiteln Leerseiten erzeugt werden. Einstiegspunkt für die Bachelorarbeit ist die Datei \texttt{abschlussarbeit.tex}. Tragen Sie zunächst den Titel der Arbeit und alle Namen ein. Bei hochschulinternen Bachlorarbeiten fallen die Eintragungen \textit{durchgeführt am} und \textit{Betreuer in der Firma} natürlich weg.