\documentclass[11pt,a4paper,twoside,openright]{memoir}

%%%%%%%%%%%%%%%%%%%%%%%%%%%%%%%%%%%%%%%%%%%%%%%%%%%%%%%
%  In my opinion (DW) there are no fonts available in the standard
%  TeX/LaTeX set that are ideal for this use, unless you go down to 9pt or 
%  8pt for your text face, and this is too small.  If you have Metafont you
%  should consider generating a cmr17 font at a magstep of two (about 25pt)
%  or three (about 30pt), or even more, depending on the point size of your
%  main text.  Why not go the whole hog and design some really fancy 
%  capitals from scratch!
%
%%%%%%%%%%%%%%%%%%%%% BOX ONE %%%%%%%%%%%%%%%%%%%%%%%%%
%\typein[\dropinitialfont]{Font for Dropped initial:} %
%\font\largefont \dropinitialfont                     %
%%%%%%%%%%%%%%%%%%%%%%%%%%%%%%%%%%%%%%%%%%%%%%%%%%%%%%%
%
%%%%%%%%%%%%%%%%%%%%% BOX TWO %%%%%%%%%%%%%%%%%%%%%%%%%
%\font\largefont= cmr10 scaled \magstep5              %
%\font\largefont= cmbx10 scaled \magstep5             %
%\font\largefont= cmbx17 scaled \magstep3             %
\font\largefont= cmr17 scaled \magstep5               %
%%%%%%%%%%%%%%%%%%%%%%%%%%%%%%%%%%%%%%%%%%%%%%%%%%%%%%%

% Copyright Symbole u..
\def\TReg{\textsuperscript{\textregistered}}
\def\TCop{\textsuperscript{\textcopyright}}
\def\TTra{\textsuperscript{\texttrademark}}

\def\drop#1#2{{\noindent
    \setbox0\hbox{\largefont #1}\setbox1\hbox{#2}\setbox2\hbox{(}%
    \count0=\ht0\advance\count0 by\dp0\count1\baselineskip
    \advance\count0 by-\ht1\advance\count0by\ht2
    \dimen1=.5ex\advance\count0by\dimen1\divide\count0 by\count1
    \advance\count0 by1\dimen0\wd0
    \advance\dimen0 by.25em\dimen1=\ht0\advance\dimen1 by-\ht1
    \global\hangindent\dimen0\global\hangafter-\count0
    \hskip-\dimen0\setbox0\hbox to\dimen0{\raise-\dimen1\box0\hss}%
    \dp0=0in\ht0=0in\box0}#2}

% end of new \drop command
%%%%%%%%%%%%%%%%%%%%%%%%%%%%%%%%%%%%%%%%%%%%%%%%%%%%%%%

%%%%%%%%%%%%%%%%%%%%%%%%%%%%%%%%%%%%%%%%%%%%%%%%%%%%%%%
% new \versal command
\newcommand{\versal}[1]{{\noindent
    \setbox0\hbox{\largefont #1}%
    \count0=\ht0                   % height of versal
    \count1=\baselineskip          % baselineskip
    \divide\count0 by \count1      % versal height/baselineskip
    \dimen1 = \count0\baselineskip % distance to drop versal
    \advance\count0 by 1\relax     % no of indented lines
    \dimen0=\wd0                   % width of versal
    \global\hangindent\dimen0      % set indentation distance
    \global\hangafter-\count0      % set no of indented lines
    \hskip-\dimen0\setbox0\hbox to\dimen0{\raise-\dimen1\box0\hss}%
    \dp0=0in\ht0=0in\box0}}
% end of new \versal command
%%%%%%%%%%%%%%%%%%%%%%%%%%%%%%%%%%%%%%%%%%%%%%%


%%%%%%%%%%%%%%%%%%%%%%%%%%%%%%%%%%%%%%%%%%%%%%%
% redefining the textblocksize
%%%%%%%%%%%%%%%%%%%%%%%%%%%%%%%%%%%%%%%%%%%%%%%
\setlrmarginsandblock{1in}{1.5in}{*}
\checkandfixthelayout
%%%%%%%%%%%%%%%%%%%%%%%%%%%%%%%%%%%%%%%%%%%%%%%


\usepackage{graphicx}
\usepackage{latexsym}
\usepackage{epsfig}
\usepackage[figuresright]{rotating}
\usepackage{dsfont}
\usepackage{german}
\usepackage[utf8]{inputenc}
%\usepackage[dvips]{color}
\usepackage{colortbl}
	\definecolor{darkblue}{rgb}{0,0,0.4}
	\definecolor{darkgray}{rgb}{0.1,0.1,0.2}
	\definecolor{darkred}{rgb}{0.6,0.0,0.0}
   \definecolor{darkgreen}{rgb}{0.0,0.6,0.0}
   \definecolor{orange}{rgb}{1.0,0.8,0.0}

%---- for algorithms ----
\usepackage[Algorithmus]{algorithm}
\usepackage{algorithmic}

%---- for the index ----
\usepackage{layouts}[2001/04/29]
\makeindex

\usepackage{hyperref}
\usepackage{memhfixc}
\hypersetup{colorlinks=true,linkcolor=darkgray,urlcolor=darkblue,citecolor=darkblue,filecolor=darkgreen}

%---- a second bibliography ----
\usepackage{multibib}
\renewcommand{\refname}{Allgemeine Literaturquellen}

%---- some new math symbols ----
\usepackage{nicefrac} % schräge bruchstriche
\usepackage{amssymb}
\newcommand{\Real}{\mathbb R}

%---- memoir package specialties
%\pagestyle{companion}
\pagestyle{Ruled}
\setlength{\epigraphwidth}{7cm}
\setcounter{tocdepth}{3}
\setcounter{secnumdepth}{3}
\maxsecnumdepth{subsection}

% change standard font
%\usepackage{cmbright}
\renewcommand{\rmdefault}{cmss}
\renewcommand{\sfdefault}{cmss}

% set figure resp. table name and number bold
\makeatletter
\renewcommand{\fnum@figure}{\textbf{\figurename~\thefigure}}
\renewcommand{\fnum@table}{\textbf{\tablename~\thetable}}
\makeatother

% setting the authors for a verse
\newcommand{\attrib}[1]{%
	\nopagebreak{\raggedleft\footnotesize #1\par}}

%---- general information ----
\author{ 
	\small{vorgelegt von} \vspace{.5cm} 											\\ 
	\Large{\textbf{Lukas Michael Müller}} 											\\ 
	\small{geb.\ in Schotten}}
\title{
	\textbf{\vspace{-2.5cm}
		%--------------------------------------------------------------
		% unter mac os x können Sie auch ein .png als Bild einbinden...
		%--------------------------------------------------------------
		\includegraphics[width=16.0cm]{thm.png} \\
		\vspace{3cm} Technologieanalyse mit Angular und WebGL zur Umsetzung eines Konfigurators für Mehrwegbecher} \\
	\vspace{1cm}
	\normalsize{Studiengang Medieninformatik} \vspace{1cm} \\
	\Large{\textbf{Bachelorarbeit}} }
\date{}


\begin{document}
	%---- the frontmatter ----
	\frontmatter
	
		\begin{titlingpage}
			\begin{center}
				%---- create the title page ----
				\maketitle
				\vspace{-1cm}
				\small{durchgeführt in bei} \\
				\small{Yellow Tree - Agentur für Kommunikation \& Design, Siegen}
				\begin{tabular}{ll}
					& \\
					& \\
					& \\
					Referent der Arbeit: & Prof. Dr. Benjamin Einert \\
					Korreferent der Arbeit: & Prof. Dr. Stefan Euler \\
					Betreuer bei Yellow Tree: & Dipl.-Ing. Andreas Utsch \\
					& \\
				\end{tabular}
				
				\vspace{\fill}
				\small{Friedberg, 2019}
			\end{center}
			
		\end{titlingpage}
	
	%---- Widmung ----
	\thispagestyle{empty}
	\begin{flushright}
		\vspace*{\fill}
		\Large{\textsl{\rmfamily{F"ur Yellow Tree}}}
		\vspace{15cm}
	\end{flushright}

	\cleardoublepage
	\setcounter{page}{1}
	
	%---- Danksagung ----
	\include{acknow/acknow}

	%---- Selbststädigkeitserkläung ----
	\chapter{Selbstständigkeitserklärung}
\label{cha:erklaerung}
%
%
Ich erkläre, dass ich die eingereichte Bachelorarbeit selbstständig und ohne fremde Hilfe verfasst, andere als die von mir angegebenen Quellen und Hilfsmittel nicht benutzt und die den benutzten Werken wörtlich oder inhaltlich entnommenen Stellen als solche kenntlich gemacht habe. \\
%
\vspace{.2cm} \\
\noindent Friedberg, Monat 2014 \\
%
\vspace{2cm} \\
\noindent Kevin-Horst Bratzke \\
%
%
%

	
	\cleardoublepage
	
	%---- table of contents ----
	\tableofcontents
	
	\cleardoublepage
	
	%---- list of figures ----
	\listoffigures
	
	%---- the mainmatter ----
	\mainmatter

		%---- Einführung ----
		%
%%%%%%%%%%%%%%%%%%%%%%%%%%%%%%%%%%%%%%%%%%%%%%%%%%%
%
% E I N L E I T U N G
%
%%%%%%%%%%%%%%%%%%%%%%%%%%%%%%%%%%%%%%%%%%%%%%%%%%%
%
\chapter{Einleitung}
\label{cha:introduction}
%
%
Mit dem folgenden Kapitel soll eine Einführung in das Thema gegeben werden. Es wird darauf eingegangen, warum dieses Thema relevant ist. Außerdem wir das Problem beschrieben, sowie das Ziel festgelegt. Am Ende des Kapitels wird auf den Aufbau der Arbeit eingegangen.
%
%
%%%%%%%%%%%%%%%%%%%%%%%%%%%%%%%%%%%%%%%%%%%%%%%%%%%
%
% Motivation
%
%%%%%%%%%%%%%%%%%%%%%%%%%%%%%%%%%%%%%%%%%%%%%%%%%%%
%
\section{Motivation}
\label{sec:motivation}
%
Der Kunde \textit{Gizeh\footnote{\textbf{Gizeh Verpackungen} ist eine international tätige deutsche Unternehmensgruppe die Kunststoffverpackungen entwirft, fertigt und dekoriert. Das Unternehmen, dessen Stammsitz sich im nordrhein-westfälischen Bergneustadt befindet, beschäftigt gegenwärtig insgesamt etwa 750 Mitarbeiter und erwirtschaftete 2017 einen Umsatz von etwa 120 Millionen Euro.}} bedruckt individuelle Mehrwegbecher. Große Druckportale, wie \textit{Fyleralarm\footnote{Die \textbf{flyeralarm GmbH} ist eine Online-Druckerei mit Sitz in Würzburg. Das Unternehmen ist auf die Herstellung und den Vertrieb von Drucksachen spezialisiert. Das Druckereiunternehmen gehört zu 100 Prozent dem Gründer Thorsten Fischer und ist in 15 europäischen Ländern präsent.}}, senden ihre Aufträge an \textit{Gizeh}. Dieser produziert dann die Becher, sowie den späteren Druck. Ein webbasierter Konfigurator könnte der Firma ermöglichen zukünftig Direktbestellungen aufzunehmen. Damit hätte der Kunde eine einfache Möglichkeit sein Design in einer 3D Ansicht zu sehen und ganz einfach zu konfigurieren. So hat man bevor der Becher bestellt wird schon einmal gesehen, wie es aussehen wird. \\
Schon in anderen Branchen, wie zum Beispiel auf dem Automarkt, wird dieses Prinzip der 3D Konfiguratoren angewandt. 
%
%
%%%%%%%%%%%%%%%%%%%%%%%%%%%%%%%%%%%%%%%%%%%%%%%%%%%
%
% Problemstellung
%
%%%%%%%%%%%%%%%%%%%%%%%%%%%%%%%%%%%%%%%%%%%%%%%%%%%
%
\section{Problemstellung}
\label{sec:problemstellung}
%
Wenn man einen Mehrwegbecher beispielsweise über \textit{Flyeralarm} bedrucken lassen will, sollte ein fertiges Design vorliegen. Dabei gibt es keine speziellen Vorgaben bzgl. der kubischen Form. Man kann lediglich ein Design im \texttt{PDF}-Format hochladen. Nebenbei wird man darauf hingewiesen, das aufgrund der kubischen Form das Design gestaucht wird. Somit werden beispielsweise Kreise eventuell nicht ganz rund dargestellt. \\ 
%
Dem Kunden wäre es somit ein Vorteil, eine Vorschau des bedruckten Bechers anschauen zu können. Die Lösung könnte ein 3D Konfigurator für Mehrwegbecher sein. Es gibt schon einige 3D Konfiguratoren. Allerdings besteht bei diesem die Frage: Wie kann der designte Becher optimal dargestellt werden? Und wie kann eine eventuell notwendige Anpassung des Designs erfolgen? Gerade wenn man den Aspekt der Responsivität hinzunimmt findet man keine bestehende Lösung, die zufriedenstellend wäre. Dieser Aspekt wird in Grundlagen Kapitel 3 unter dem Punkt \ref{sec:responsive} \textit{Responsive Webdesign} genauer erläutert.
%
%
%%%%%%%%%%%%%%%%%%%%%%%%%%%%%%%%%%%%%%%%%%%%%%%%%%%
%
%  Zielsetzung
%
%%%%%%%%%%%%%%%%%%%%%%%%%%%%%%%%%%%%%%%%%%%%%%%%%%%
%
\section{Zielsetzung}
\label{sec:zielsetzung}
%
Im Rahmen der Arbeit wird ein 3D Konfigurator für Mehrwegbecher entwickelt. Dazu wird das Framework \textit{Angular} und \textit{WebGL} verwendet\footnote{Es wird später noch genauer auf die Umsetzung eingegangen. Im Grundlagenkapitel werden die einzelnen Technologien genauer beleuchtet}. Er soll die Funktionalität haben, ein Design in einer 3D Vorschau auf einen Becher zu rendern. Eine optionale Funktion wäre das Anpassen des Designs durch zuschneiden oder transformieren. Dabei soll der Konfigurator auch auf mobilen Geräten verwendbar sein. Es ist wichtig das die Anwendung schnell und einfach zu bedienen ist.
%
%
%%%%%%%%%%%%%%%%%%%%%%%%%%%%%%%%%%%%%%%%%%%%%%%%%%%
%
%  Vorgehen bei der Umsetzung
%
%%%%%%%%%%%%%%%%%%%%%%%%%%%%%%%%%%%%%%%%%%%%%%%%%%%
%
%
\section{Vorgehen bei der Umsetzung}
\label{sec:vorgehen}
%
\paragraph{Entwicklungsumgebung}Wie schon erwähnt wird \textit{Angular} verwendet. Das Framework eignet sich besonders gut, da hiermit die Anwendung gut testbar und wartbar umsetzbar ist. Bei der Entwicklung werden die Stärken und Schwächen des Frameworks aufgezeigt. Es wird versucht eine möglichst performante Anwendung zu programmieren. Als Editor wird \textit{PhpStorm\footnote{https://www.jetbrains.com/phpstorm}} verwendet. Er bietet eine gute Implementierung von Angular Projekten und hat weitere nützliche Funktionen, die das Entwickeln vereinfachen.
%
\paragraph{Design}Da der Konfigurator später in eine bestehende Webseite eingebaut werden soll\footnote{Dies ist nicht Bestandteil der Arbeit. Zuerst muss Kunde dem Projekt noch zustimmen. Anschließend müssen noch eventuelle Änderungen vorgenommen werden.}, wird sich das Design am Stil der Webseite orientieren. Genauere Vorgaben dazu gibt es nicht. Deshalb wird eine Aufgabe sein ein innovatives Design zu finden. Es soll den Ansprüchen der \textit{Usability} gerecht werden.
%
\paragraph{3D Szene}
Nahezu jeder Browser unterstützt WebGL. Die \textit{three.js} Library bietet dem Entwickler eine Abstraktionsschicht über \textit{WebGL}, um benötigte 3D Szenen bauen zu können. Mithilfe dieser JavaScript Library wird der Becher mit dem hochgeladenen Design gerendert. Auf dieses Thema wird in Punkt \ref{sec:javascriptbibliotheken} \textit{JavaScript Bibliotheken} noch einmal Bezug genommen.
%
\paragraph{Testumgebung}
Am Ende der Entwicklung wird die entwickelte Applikation mit der Testumgebung von Angular untersucht und anschließend ein Fazit daraus gezogen. Das Framework liefert von Haus aus eine gute Möglichkeit um Unit-Tests sowie End-to-End-Tests durchzuführen.
%
%
%%%%%%%%%%%%%%%%%%%%%%%%%%%%%%%%%%%%%%%%%%%%%%%%%%%
%
%  Aufbau der Arbeit
%
%%%%%%%%%%%%%%%%%%%%%%%%%%%%%%%%%%%%%%%%%%%%%%%%%%%
%
\section{Aufbau der Arbeit}
\label{sec:aufbau}
%
In dem Kapitel \textit{2 Stand der Technik} wird zunächst auf vorhandene Lösungen bzw. Lösungsansätze eingegangen. Es soll deutlich werden, warum diese nicht ausreichend sind. Außerdem werden Beispiele aus anderen Branchen angeführt, die eine gute 3D Konfiguration für andere Produkte ermöglichen. Kapitel \textit{3 Grundlagen} erklärt grob die verwendeten Technologien. Durch Beispiele wird gezeigt, warum diese sinnvoll zur Umsetzung eines 3D Konfigurators sind. Es werden auch einige wichtige Begriffe erläutert, die wichtig zur Realisierung von modernen Webanwendungen sind.\\
%
Kapitel \textit{4 Anforderungen und Konzeption} beschreibt die Entwicklung des Projektes. Nachdem das Problem noch einmal genauer analysiert wird und die Anforderungen an den Konfigurator klar sind, soll anschließend das Konzept mithilfe von Diagrammen und Mockups erläutert werden. 
Dann wird die Programmierung des Konfigurators gezeigt und wie es zur Endlösung kam. Dabei wird auf wesentliche Bestandteile und Funktionen der Anwendung genauer eingegangen. Der Leser soll verstehen warum es zu bestimmten Teillösungen kam und wie die Technologien Angular und WebGL dazu verwendet werden konnten. \\
Zuguterletzt beschäftigen sich die Kapitel \textit{6 Ergebnisse} und \textit{7 Zusammenfassung} mit den Evaluation der Anwendung. Zunächst werden einige Unit- und End-to-End-Tests durchgeführt. Danach wird das Ergebnis der Tests erläutert und infolgedessen schließlich ein Fazit gefasst.
%

		%---- Kapitel Stand der Technik ----
		%
%%%%%%%%%%%%%%%%%%%%%%%%%%%%%%%%%%%%%%%%%%%%%%%%%%%
%
%  S T A N D   D E R   T E C H N I K
%
%%%%%%%%%%%%%%%%%%%%%%%%%%%%%%%%%%%%%%%%%%%%%%%%%%%
\chapter{Stand der Technik}
\label{cha:technik}
%
%
Im Folgenden werden vergleichbare Arbeiten, Projekte und deren Einsatzgebiete angeführt. Es wird zunächst auf vergleichbare Projekte und anschließend auf die in der Arbeit verwendeten Technologien eingegangen. Dabei wird analysiert, warum eine Technologieanalayse im Kontext dieser Arbeit sinnvoll ist.
%
%
%
%%%%%%%%%%%%%%%%%%%%%%%%%%%%%%%%%%%%%%%%%%%%%%%%%%%
%
% A U F B A U 
%
%%%%%%%%%%%%%%%%%%%%%%%%%%%%%%%%%%%%%%%%%%%%%%%%%%%
%
\section{Bestehende Lösungsansätze}
\label{sec:vergleiche}
%
\subsection{Flyeralarm}
\label{sec:flyeralarm}
%
Auf der Internetseite \textit{Flyeralarm} gibt es nicht nur die Möglichkeit Flyer zu drucken. Es gibt sehr viele Kategorien mit einigen Produkten. Eine der Kategorien sind Becher. Man hat auch die Möglichkeit Mehrwegbecher zu bedrucken (siehe Abbildung \ref{fig:flyeralarm}).
%
\begin{figure}
	\centering
	{\epsfig{file = technik/images/flyeralarm.png, width=11.0cm}}
	\caption[Mehrwegbecher bestellen]{\textit{Bestellvorgang von Mehrwegbechern auf flyeralarm.com}}
	\label{fig:flyeralarm}
\end{figure}
% 
Dazu muss man, nachdem man die gewünschte Größe gewählt hat, sein Design hochladen. Dabei sollte man das Datenblatt\footnote{Das Datenblatt ist auf der Webseite zu finden und kann von dort heruntergeladen werden.} berücksichtigen, welches die Vorgaben beschreibt wie zum Beispiel Sicherheitsabstand oder Druck Farbraum. Die Online-Druckerei schreibt auf ihrer Webseite:\textit{\glqq Aus produktionstechnischen Gründen und aufgrund der konischen Form der Becher wird das Layout beim Druck in der Breite gestaucht. Davon sind vor allem grafische Elemente wie Kreise betroffen.\grqq} \cite{flyeralarm_mehrwegbecher_nodate} Eine Vorschau, wie das ganze am Ende aussehen wird, gibt es nicht. Im schlechtesten Fall hat man am Ende ein nicht zufriedenstellendes Ergebnis. Auch andere Seiten bieten ein ähnliches Angebot.Eine fertige Vorschau ist jedoch eher nicht zu finden.
%
\subsection{Spread Shirt}
\label{sec:spreadshirt}
%
Spreadshirt ist eigentlich eine Onlinedruckerei für T-Shirts. Sie selbst schreiben über sich Folgendes: \textit{\glqq Seit 2002 liefert Dir Spreadshirt T-Shirt-Druck in bester Qualität. Was als Start-up-Idee in Leipzig begann, ist inzwischen ein weltweit erfolgreiches Print-on-Demand-Unternehmen, das Wert auf faire Handels- und Produktionswege legt, seine Verantwortung als internationaler Arbeitgeber ernst nimmt und seinen Mitarbeitern ein attraktives Arbeitsumfeld bietet.\grqq } \cite{spreadshirt_tassen_nodate}
%
\begin{figure}[h]
	\centering
	{\epsfig{file = technik/images/spread-shirt.png, width=11.0cm}}
	\caption[Tasse bestellen]{\textit{Gestaltung einer Tasse auf spread-shirt.com}}
	\label{fig:spread}
\end{figure}
%
Zwar wird kein Druck von Mehrwegbechern angeboten, dafür aber der Druck von Tassen. Der Kunder kann hier in einem Online-Konfigurator seine Tasse selbst gestalten, indem er Text oder Bildelemente auf die Tasse legt. Das ganze wird sogar in 3D angezeigt. Jedoch ist die Ansicht nicht flexibel sondern statisch. Man kann die Tasse lediglich von drei verschiedenen Blickwinkeln betrachten (rechts, vorne, links). In unserem Fall sollen fertige Designs dargestellt werden. Auch das ist beim Tassenkonfigurator von Spreadshirt schwierig. Wie in der Abbildung \ref{fig:spread} zu sehen kann ein Objekt nur im Druckbereich dargestellt werden, nicht außerhalb des Bereichs. Damit ist ein Rundum-Druck nicht möglich. \\
Trotzdem lässt sich sagen, dass dieser Konfigurator gut und übersichtlich gestaltet ist. Jedoch hat er nicht die Funktionalität, welche der Konfigurator für Gizeh haben sollte.
%
\subsection{Becher-bedrucken.de}
\label{sec:currycup}
%
Ein weiteres Angebot für Becher gibt es auf \textit{becher-bedrucken.de}. Dort gibt es einen 3D Konfigurator für verschiedene Becher. Der technische Ansatz ist schon sehr gut und kann bei der Entwicklung berücksichtigt werden. Es werden ähnliche Technologien verwendet, wie in dieser Arbeit. Jedoch sind die Designvorgaben ganz anders als die Druckvorgaben von Gizeh. Das hochgeladene Design ist so angepasst, das es bestmöglich auf dem Becher angezeigt werden kann.
%
\begin{figure}[h]
	\centering
	{\epsfig{file = technik/images/becher-bedrucken.png, width=11.0cm}}
	\caption[Becher bedrucken]{\textit{3D Viewer eines Bechers auf bedrucken-becher.de}}
	\label{fig:becherbedrucken}
\end{figure}
%
Die Darstellung in 3D ist schön anzusehen. Zumindest auf einem Gerät mit größerem Display. Eine Anpassung für mobile Geräte ist so gut wie gar nicht vorhanden. Technisch gesehen kann aber trotzdem an diesen Lösungsansatz angeknüpft werden.
%
\section{3D Online Konfiguratoren}
\label{sec:3dconfigurators}
%
Heutzutage kann nahezu alles bedruckt werden. Die Vielzahl an Produkten ist groß. Dies haben wir bisher in diesem Kapitel erläutert, wie das konkret bei Bechern oder Tassen aussehen kann. Schwieriger zu finden sind allerdings 3D Konfiguratoren für diese Produkte. Oft gibt es höchsten eine 3D Pop-Up Ansicht, eine alte Lösung mit Flash o. ä. Eine responsive Lösung ist eher nicht zu finden. \\
In der Autoindustrie und anderen Branchen sind jedoch einige gute 3D Konfiguratoren umgesetzt. Wenn man eventuell die passenden Felgen sucht, bekommt man da einen übersichtlich gut gestalteten Konfigurator in 3D. Teilweise sind Konfiguratoren zu finden, welche responsiv sind. Einige Firmen bieten sogar an 3D Konfiguratoren für bestimmte Produkte umzusetzen. Dabei handelt es sich meist um individuelle Lösungen. Das Folgende Beispiel soll zeigen, das es durchaus möglich ist gute 3D Konfiguratoren zu entwickeln.
%
\begin{figure}[]
	\centering
	{\epsfig{file = technik/images/audi.png, width=6.0cm}}
	\caption[Audi Konfigurator]{\textit{Neue 3D Ansicht von Audi}}
	\label{fig:audi}
\end{figure}
%
%
\paragraph{Audi 3D Konfigurator}Letztes Jahr veröffentlichte Audi seinen neuen Konfigurator auf der Webseite. Er rendert die Fahrzeuge in Echtzeit. Die Anwendung ist auch für mobile Geräte optimiert (siehe Abbildung \ref{fig:audi}). So kann man den Konfigurator beispielsweise mit Toucheingaben steuern. Man kann sich das Fahrzeug ganz genau anschauen und an Details heranzoomen. Technisch eine sehr gute Umsetzung, die auch optisch etwas her macht. Für die Zukunft plant Audi sogar eine 4k-Darstellung. Dardurch bekommt der Benutzer eine noch realistischere Ansicht des Wagens. \\
%
%
\paragraph{Siemens 3D Konfigurator}Immer mehr Firmen präsentieren neue 3D Konfiguratoren für ihre Produkte. So auch 2018 Siemens. Mit einem 3D Konfigurator ist es nun möglich Systemschaltschränke am 3D-Modell zu konfigurieren. Dabei wählt der Nutzer die verschiedenen Module, welche in den Schaltschrank verbaut werden sollen, in mehreren Schritten aus. Der Konfigurator prüft dabei, ob eine Kombination der Module möglich ist. Sogar eine Exportfunktion für CAD-Programme ist in den Konfigurator integriert. Die Kosten der Zusammenstellung werden auch in einer übersicht dargestellt.\\
%
\paragraph{Trilux Limba 3D Konfigurator}Immer mehr Firmen präsentieren neue 3D Konfiguratoren für ihre Produkte. So auch 2018 Siemens. Mit einem 3D Konfigurator ist es nun möglich Systemschaltschränke am 3D-Modell zu konfigurieren. Dabei wählt der Nutzer die verschiedenen Module, welche in den Schaltschrank verbaut werden sollen, in mehreren Schritten aus. Der Konfigurator prüft dabei, ob eine Kombination der Module möglich ist. Sogar eine Exportfunktion für CAD-Programme ist in den Konfigurator integriert. Die Kosten der Zusammenstellung werden auch in einer übersicht dargestellt.\\
%
\section{Webframeworks}
\label{sec:webframeworks}
%
Heutzutage werden Frameworks wie \textit{Angular} oder \textit{ReactJS} verwendet um performante und benutzerfreundliche Webanwendungen zu entwickeln. Es handelt sich dabei um JavaScript-Frontend-Frameworks. Man hat oft die Qual der Wahl, da in den letzten Jahren einige Frameworks entwickelt bzw. weiterentwickelt wurden. Die Anzahl der Angebote an JavaScript-Frameworks und -Libraries ist groß. Sie kommen oft beim Entwickeln von \textit{Single Page Applications (SPA)}\footnote{Was Single Page Anwendungen sind wird im Kapitel Grundlagen genauer beschrieben.}  zum Einsatz.
%
\paragraph{Angular}
\label{p:angular}
%
ist nichts anderes als ein JavaScript-Framework auf Basis von TypeScript\footnote{Eine von Microsoft entwickelte Programmiersprache. TypeScript ist eine kompilierte und plattformübergreifende Sprache, die reine JavaScript-Dateien generiert.}. Es wurde von Google entwickelt und ist ein Open-Source-Framework. Es unterstützt den Entwickler dabei, moderne Webanwendungen zu machen, die zum einen für Desktop und zum andern für Mobile optimiert worden sind. Wie in der Abbildung \ref{fig:googletrends} zu sehen, ist \textit{Angular} das meist genutzte und bekannteste Framework für SPAs. Ähnliches zeigt auch die Stack Overflow Entwicklerumfrage 2018: Bei den meist genutzen Bibliotheken und Frameworks liegt Angular mit 36,9\% einen Platz vor React mit 27,8\% \cite{stackoverflow_stack_2018}.
%
\begin{figure}[h]
	\centering
	{\epsfig{file = technik/images/google-trends.png, width=13.0cm}}
	\caption[Audi Konfigurator]{\textit{Google Trends Statistik. Suchanfragen von Frontend-Frameworks in 2018}}
	\label{fig:googletrends}
\end{figure}
%
Außerdem gibt es den Entwicklern Konventionen und Richtlinen an die Hand. Das ist gerade dann sinnvoll, wenn man in einem großen Team arbeitet. Oder bei großen Projekten für ein einfaches Handling und die Wartbarkeit des Codes sinnvoll wird. Ein Merkmal von Angular ist, das es viele unterschiedliche Module gibt, die es ermöglichen, sehr effizient Anwendungen zu produzieren. Da gibt es Module, welche für die Client-Server-Kommunikation zuständig sind. Sie ermöglichen die Kommunikation mit dem Backend. Ein anderes Modul ist für die Bindung der Daten zuständig. Wenn ich auf einer Ansicht Informationen bereitstellen möchte, die zum Beispiel aus einer Datenbank kommen, kann ich das ganz einfach mit dem Bindungsmechanismus von Angular umsetzen. Das Framework ist auch bekannt für seine tollen Animationen auf Basis von Web-Animation. Ein weiteres, wichtiges Modul ist das Routing-Modul. Im Grunde genommen sind Routings Grundbestandteil für \textit{Single Page Anwendungen}. Mittels Routings kann ich festlegen welcher Teil der Anwendung nun angezeigt werden soll. Außerdem baut das Framework auf die komponentenbasierte Programmierung. In Kapitel 3 werden wir noch genauer auf Angular eingehen\footnote{Die Dokumentation des Frameworks ist unter \textit{https:/angular.io/docs/} zu finden.}.
%

%
\paragraph{React}
\label{p:react}
%
ist das meistgesuchte JavaScript Framework 2018 \cite{stackoverflow_stack_2018}. Es wurde 2013 von Facebook entwickelt und viele bekannte Unternehmen wie Netflix, Twitter oder PayPal verwenden das Framework. Ähnlich wie Angular setzt React auf modulare Komponentenarchitektur. Damit wird der Frontendcode leicht nachvollziehbar. \textit{\glqq Das Ziel von React ist es, einfacheren Code schreiben zu können, dessen Bestandteile weniger miteinander verschränkt oder verwoben sind \grqq}\cite{kogel_paul_react_2015} \\
React ist kein Framework, es ist eine Bibliothek. Es ist ein sehr flexibles Werkzeug, dass in bestehende Anwendungen eingebaut werden kann, ohne den ganzen Code umzustrukturieren. Dem Entwickler wird keine Grundstruktur für seine Applikation gegeben, was ein wesentlicher Unterschied zu Angular ist. Ein weiteres Merkmal von React ist die virtuelle DOM. Diese garantiert die Synchronisierung der DOM indem bei Änderungen alles neu gerendert wird\footnote{Die vollständige Dokumentation ist unter \textit{https://reactjs.org/docs/getting-started.html} zu finden}.
%
\paragraph{Vue.js}
\label{p:vueJS}
%
ist ein weiteres Frontend-Framework zur Entwicklung von \textit{Single Page Anwendungen}. Es ist nicht das beliebteste Framework und trotzdem taucht es immer wieder auf (siehe Abbildung \ref{fig:googletrends}) und ist definitiv eine Alternative zu \textit{Angular} oder \textit{React}. Es setzt auch auf eine modulare Architektur, die in einzelne Komponenten zerteilt ist. Das Framework ist im Vergleich zu seinen Konkurrenten deutlich einfacher und hat eine flache Lernkurve. Dadurch ist es auch schneller und bringt dem Framework einen Vorteil gegenüber den Alternativen. Obwohl es kleiner ist, bringt es trotzdem alle wichtigen Funktionen wie Suchmaschinenoptimierung mit sich. Es kann auch in Kombination mit React verwendet werden. In Vue.js wird ohnehin das bekannte virtul DOM genutzt. Auch die Community ist stetig am wachsen.\footnote{Weitere Informationen sind in der offizielen Dokumentation unter \emph{https://vue.js.org/v2/guide} zu finden.} \\
%
%
%
%%%%%%%%%%%%%%%%%%%%%%%%%%%%%%%%%%%%%%%%%%%%%%%%%%%
%
% A U F B A U 
%
%%%%%%%%%%%%%%%%%%%%%%%%%%%%%%%%%%%%%%%%%%%%%%%%%%%
%
\section{Flash und WebGL}
\label{sec:webgl2}
%
In dem Artikel \emph{HTML5/WebGL vs Flash in 3D Visualisation} schreibt der Autor folgendes über WebGL:\\

\textit{\glqq The development of improved 3D graphics in Web-based applications took a step forward recently, when programmers began building WebGL into the Mozilla Firefox nightly builds, and into WebKit, which is used by Google Chrome and Apple's Safari browser. WebGL is one of the most developed libraries which are supported by HTML5.\grqq } \cite{bahor_html5/webgl_2013}.\\

Adobe Flash ist auch heute noch vielen ein Begriff. Es war darauf ausgerichtet interaktive 2D Grafiken im Web bereitzustellen. Mit der Veröffentlichung der Version 10 des Flash Players haben die Entwickler sogar eine z-Achse eingeführt. Sie ermöglicht eine 3D Darstellung und Transformation von Objekten. Diese Unterstützung für die Interaktion von Objekten der dritten Dimension wurde jedoch in begrenzter Weise bereitgestellt.\\
Einer der aktuellsten Leistungstests zwischen dem Flash- und dem WebGL-Canvas-basierten 3D-Inhalt zeigt deutlich, dass der HTML5-Canvas beginnt, höhere Frameraten zu generieren und 3D-Inhalte im Web zu rendern. Die Testergebnisse zeigen, dass WebGL beim Rendern von 3D-Inhalten wesentlich schneller abschneidet und höhere Frameraten für die 3D-Animationen im Web bietet, während sie mit Flash verglichen werden. Weiter schreibt Senad Bahor in seinem Artikel: \textit{\glqq And actually, that ist what the 3D graphics is all about, about altering the 3D model abd see it change on time basis. This is something that can currently be achieved only with the usage of the HTML5 and WebGL engines on the web [...] \grqq} \cite{bahor_html5/webgl_2013} 3D-Modelle zu ändern und sie zu animieren kann also derzeit nur durch die Verwendung von HTML5 und WebGL im Web erreicht werden.\\
Weiter zeigen die Ergebnisse des Artikels deutlich, dass Flash-basierte 3D-Grafiken auf iOS-basierten Geräten nahezu nicht darstellbar sind, da Adobe mit seinem Flash-Plugin nie iOS unterstützt hat. Aufgrund der HTML5-Funktionalität und -Erreichbarkeit kann die Webseite auch auf einer Vielzahl von Geräten, einschließlich iOS- und Android-Geräten, wiedergegeben werden, ohne dass sich die Benutzer um das Vorhandensein der Plugins und die Versionierung des Players kümmern müssen.
Webbrowser entwickeln sich stetig weiter. WebGL kann in über 95\% aller Browser verwendet werden \cite{deveria_alexis_can_2013}. Was zuvor schon in Videopielen mit OpenGL möglich war, ist nun auch im Web möglich. Wie genau WebGL als Abstraktionsschicht für den grafischen Teil einer Anwendung verwendet werden kann, wird in Kapitel \ref{sec:javascriptbibliotheken} erläutert.

Mit WebGL müssen die Benutzer nicht dazu aufgefordert werden, eines der Plugins zu installieren, um den 3D-Inhalt auf der Website zum Laufen zu bringen. Die einzige Voraussetzung ist, dass der Webbrowser das HTML5-Canvas-Element unterstützt, über das WebGL den Inhalt für den Benutzer darstellt.\\
Im Wesentlichen kann WebGL auf jeder Plattform und auf allen großen Systemen mit OpenGL-fähiger Grafikkarte und einem Browser ausgeführt werden, der WebGL unterstützt.\\

\section{3D JavaScript Bibliotheken}
\label{sec:javascriptbibliotheken}
%
Um eine 3D Szene mit WebGL darzustellen wird ein JavaScript Framework benötigt. Es erstellt und rendert eine Szene mit den 3D Objekten. Im Folgenden werden die zwei bekanntesten und weit verbreitetsten Bibliotheken vorgestellt.
\paragraph{ThreeJS}
\label{sec:threeJS}
%
\textit{Three.js} ist es eine sehr seriöse WebGL-Bibliothek mit einer starken Community und vielen guten Beispielen. Es wird auch oft in kommerziellen Webanwendungen verwendet. Die erste Version des Frameworks tauchte 2010 auf und der Quellcode wird in einem Repository auf GitHub gehostet\footnote{https://github.com/mrdoob/three.js/}. Three.js hat eine verständliche Struktur und große Anpassungsmöglichkeiten. Es ist eine Anwendungsprogrammierschnittstelle, mit der animierte 3D-Grafiken in einem Webbrowser erstellt und angezeigt werden. Neben der Dokumentation gibt es auch ein Wiki, was dem Entwickler zum schnellen Einstieg sehr helfen kann\footnote{https://github.com/mrdoob/Three.js/wiki/}.Schon beim erstellen einer einfachen Szene mit dem Framework fällt auf, das es dem Prinzip der objektorientierten Programmierung folgt\footnote{https://threejs.org/docs/\#manual/en/introduction/Creating-a-scene}. In dem Kapitel \ref{cha:introduction} wird noch genauer auf das Framework eingegangen.
%
%\begin{figure}[h]
%	\centering
%	{\epsfig{file = technik/images/node-map.png, width=11.0cm}}
%	\caption[Audi Konfigurator]{\textit{Aufbau einer Szene in Three.js}}
%	\label{fig:threejs}
%\end{figure}
%
%
\paragraph{BabylonJS}
\label{sec:babylonJS}
%
Babylon.js\footnote{Die offizielle Dokumentation ist unter \textit{https://doc.babylonjs.com/} zu finden.} ist ein Framework, mit dem komplette 3D-Webanwendungen erstellen werden können. Babylon.js hat eine Community, die stetig wächst und auch aktiv zum Projekt beiträgt und immer mehr Funktionen hinzufügt. Das Framework hat alle notwendigen Werkzeuge, um 3D-Anwendungen umzusetzen. Sie können 3D-Objekte laden und zeichnen, diese 3D-Objekte verwalten, Spezialeffekte erstellen und verwalten, räumliche Sounds spielen und verwalten, Gameplay erstellen und vieles mehr. 
%
\begin{figure}[h]
	\centering
	{\epsfig{file = technik/images/3d-framework.png, width=13.0cm}}
	\caption[Audi Konfigurator]{\textit{Google Trends: Vergleich Three.js und Babylon.js in den letzten 12 Montaten}}
	\label{fig:compare3dframework}
\end{figure}
%
Babylons.js ist ein benutzerfreundliches Framework, da Sie diese Dinge mit den minimalen Codezeilen einrichten können. Es ist ein mit TypeScript entwickeltes JavaScript-Framework. (vgl. \cite{moreau-mathis_babylon.js_2016}) Wie in Abbildung \ref{fig:compare3dframework} zu sehen ist Babylon.js jedoch weniger gefragt als three.js. Das zeigt sich auch in den 3D Webanwendungen, wo auch meist three.js verwendet wird.

		
		%---- Kapitel Grundlagen ----
		%
%%%%%%%%%%%%%%%%%%%%%%%%%%%%%%%%%%%%%%%%%%%%%%%%%%%
%
%  E N T W I C K L U N G S U M G E B U N G
%
%%%%%%%%%%%%%%%%%%%%%%%%%%%%%%%%%%%%%%%%%%%%%%%%%%%
\chapter{Grundlagen}
\label{cha:grundlagen}
%
%
In dem Grundlagenkapitel geht es um das Basiswissen, auf dem die Arbeit aufbaut. Es wird näher auf die verwendeten Technologien eingegangen. Dabei werden die Frontend-Frameworks beleuchtet sowie die 3D Bibliotheken. Näher wir auch auf die Begriffe Resposive Webdesign, Usability und Performance eingegangen.
%
%
%%%%%%%%%%%%%%%%%%%%%%%%%%%%%%%%%%%%%%%%%%%%%%%%%%%
%
% S P A
%
%%%%%%%%%%%%%%%%%%%%%%%%%%%%%%%%%%%%%%%%%%%%%%%%%%%
%
\section{Single Page Anwendungen}
\label{sec:spa}
%
Wofür es früher Flash gab, gibt es jetzt Frameworks wie Angular oder Vue.Js. Es stellt sich aber die Frage, was genau eigentlich Singe Page Applications (SPA) sind. Im Folgenden wollen wir uns dies etwas genauer anschauen.
\paragraph{Klassische Webanwendungen}
Bei einer klassischen Webanwendung ist Logik von Design und Style getrennt. Es gibt für jede Seite einer Anwendung eine eigene Seite. Das hat den Nachteil, das alle Seiten mit Ajax\footnote{Ajax bzw. AJAX steht als Akronym für „Asynchronous JavaScript and XML“. Die Technologie ermöglicht es, einzelne Teile einer Webseite bei Bedarf asynchron zu laden, so dass sie dynamisch wird. Der angezeigte Inhalt lässt sich gezielt so manipulieren, ohne die komplette Seite neu zu laden.} geladen werden müssen und im Hintergrund abarbeitet werden. Bei User Eingaben oder Änderungen in der Seite muss immer die Seite neu geladen werden. Das kann bei komplexen Anwendungen und hohen Besucherzahlen der Seite beispielsweise zu einer Serverüberlastung führen.
\paragraph{Dezentralisierung}
%
Ein Ziel von SPAs ist es, die Kommunikation zwischen Client und Server enorm zu reduzieren. Um dies zu erreichen wird die Anwendung dezentralisiert. Das heißt, es gibt nur ein HTML Dokument\cite{domin_was_2018}. Durch den Einsatz von Frameworks, wie Angular, ist es möglich Inhalte zu aktualisieren, ohne die Seite im Hintergrund neu zu laden oder zu einer anderen Seite zu wechseln.
\footnote{Das Prinzip von Single Page Anwendungen wird zum Beispiel von Googles Gmail und Gmaps sowie von Twitter verwendet}
%
%
%%%%%%%%%%%%%%%%%%%%%%%%%%%%%%%%%%%%%%%%%%%%%%%%%%%
%
% A U F B A U 
%
%%%%%%%%%%%%%%%%%%%%%%%%%%%%%%%%%%%%%%%%%%%%%%%%%%%
%
\section{Framework Angular}
\label{sec:angular7}
Das Framework Angular ist sehr umfangreich. Im Folgenden werden einige Grundlagen erläutert, die zur Implementierung einer Anwendung mit Angular benötigt werden.
\subsection{Komponentenbasierte Programmierung}
%
Angular ist ein Framework, das auf Komponenten setzt. Das Prinzip der komponentenbasierten Entwicklung kommt nicht nur bei Angular vor, sondern auch bei vielen anderen Programmiersprachen und Frameworks. Was in Angular Komponenten sind wird in dem Buch \textit{Angular: Grundlagen, fortgeschrittene Techniken und Best Practices mit TypeScript - ab Angular 4, inklusive NativeScript und Redux} folgendermaßen beschrieben:\\
%

\textit{\glqq Komponenten sind die Grundbausteine einer Angular Anwendung. Jede Anwendung ist aus vielen verschiedenen Komponenten zusammengesetzt, die jeweils eine bestimmte Aufgabe erfüllen. Eine Komponente beschreibt somit immer einen kleinen Teil der Anwendung, z. B. eine Seite oder ein einzelnes UI-Element.\grqq }\cite{woiwode_angular:_2017} \\

%
Im Grunde ist eine Komponente also nichts anderes als ein selbstdefinierter HTML-Knoten, den man im HTML Kontext ganz normal nutzen kann. Man erzeugt einfach DOM-Elemente\footnote{Wenn eine Webseite geladen wird, erstellt der Browser eine Document Object Model der Seite. Das HTML-DOM-Modell ist als Baum von Objekten aufgebaut. Diese Objekte sind ein der Regel einfache HTML-Tags} und kann sie dann automatisch in Angular nutzen.
Eine Komponente in Angular ist in drei Parts aufgeteilt: Logik, Vorlage und Style. Die Logik beschreibt die TypeScript Klassen, Eigenschaften, Methoden und so weiter. Sie beschreibt also, was zu tun ist, wenn zum Beispiel ein bestimmter Button geklickt wird. Die Vorlage (Template) ist ein HTML-Schnipsel. Dies kann einfach nur ein Button-Element sein oder auch eine HTML-Struktur mit mehreren HTML-Elementen. Sie beschreibt die Stuktur bzw. den Aufbau der Komponente. Wie die Komponente aussehen soll wird in dem Style Part festgelegt. Dabei ist zu beachten, dass die Style-Definitionen nur innerhalb der Komponente gelten. Globale Styles der Anwendung werden seperat definiert.\\
Der Startpunkt der Anwendung ist die \texttt{index.html}. Sie definiert die Applikationskomponente, der Ankerpunkt einer Angular-Anwendung, quasi eine Basiskomponente. In Dieser können nun weitere Kindskomponenten eingefügt werden, welche auch ineinander verschachtelt sein können. So entsteht eine komplette Struktur. Eine Komponentenvorlage kann also nicht nur HTML-Elemente enthalten, sondern auch Kindskomponenten. Entscheidend ist, das Komponenten verschachtelt werden können. Dies ist das Grundprinzip der komponentenbasierten Programmierung (vgl. Angular Grundkurs\cite{unlu_angular_2018}).

\subsection{Modulare Umsetzung}
Schauen wir uns nun an, wie Angular mit Modulen arbeitet und was genau Module sind. \textit{Nikolas Poniros} hat das in seinem Buch \textit{Angular für Dummies} so definiert:\\

\textit{\glqq Angular-Module helfen beim Gliedern einer Webanwendung in verschiednene Funktionsblöcke. Alle Angular-Bausteine, die logisch zu einem Funktionsblock gehören, werden mit dem entsprechenden Angular-Modul registriert. Die registrierten Bausteine gehören dann zum Angular-Modul. Angular-Module sind von der Denkweise her vergleichbar mit ECMA-Script-Modulen. ECMA-Script-Module kapseln TypeScript-Konstrukte wie Klassen und Funktionen und erlauben nur den Zugriff auf exportierte Konstrukte. Angular-Module kapseln Komponenten, Pipes und Direktiven\footnote{Was genau es mit diesen Angular Bausteinen auf sich hat wird in Punkt \ref{subsec:grundfunktionen} beleuchtet.}. Ein anderes Angular-Modul kann nur auf einen Baustein zugreifen. wenn es dieses exportiert.\grqq \cite{poniros_angular_2019} }\\

Angular bietet also die Möglichkeit modular zu arbeiten. Die einzelnen Elemente und Funktionalitäten können in Modulen gruppiert werden. Im Grunde ist ein Modul nichts anderes als ein Container. Sie lassen sich besonders gut verwenden, wenn man mit mehreren Entwicklern zusammenarbeiten möchte. Module sind einzelne Bestandteile der Anwendung und lassen sich ganz einfach in bestehende Anwendungen einbauen. So kann ich ein Modul in mehreren Anwendungen wiederverwenden. Oder verschiedene Teammitglieder entwickeln jeweils ihre eigenen Module, welche anschließend in einer Anwendung zusammengefügt werden können. Ich kann ein Modul also auch in einem anderen Kontext wiederverwenden. Ein Modul darf allerdings nur in einem einzigen Modul initialisiert werden. Stattdessen wird das Modul bei mehrfacher Verwendung lediglich importiert. Folgendes Beispiel soll das Prinzip modularer Umsetzung verdeutlichen.\\

%
Beispiel mit Modul A und Modul B\\
%

\paragraph{Hauseigene Module}
Angular hat auch ein paar hauseigene Module, die je nach Anwendungsfall importiert werden können. Das \textit{BrowserModule} wird für Web-Anwendungen im Angular-Umfeld genutzt und verfügt über alle Funktionalitäten, mit dem wir in der Lage sind, Ereignisse innerhalb des Browsers abzufangen, DOM-Rendering durchführen zu können und damit schließlich die lauffähige Angular-Anwendung im Browser realisieren zu können.\\
Das \textit{CommonModule} beinhaltet allgemeine Funktionen, die sehr, sehr häufig genutzt werden. Diese Funktionen liegen in Form von Direktiven und Pipes vor, womit ich bestimmen kann, ob ein HTML-Knoten angezeigt wird oder nicht oder auch Pipes, womit ich Ausgaben formatieren kann. Auch sprachabhängige Funktionalitäten sind in den \textit{CommonModule} enhalten.\\
Das \textit{HttpModule} ist dafür da, Client-Server-Kommunikation zu betreiben. Das heißt, HTTP-Requests lassen sich mit Hilfe des \textit{HttpModules} hervorragend realisieren, dafür werden Services zur Verfügung gestellt. Das \textit{HttpModule} hat auch Services für Testings und Co. Für Formulare gibt es entweder das FormsModule oder das \textit{ReactiveFormsModule}. Das hängt so ein wenig davon ab, wie man Formulare in der Anwendung gestalten will.\\
Das \textit{RouterModule} ist dafür da, um Komponenten-Routing zu realisieren. Das Routing ist die Grundlage für \textit{Single-Page-Applications}. Das heißt, über das Routing kann man bestimmen, welche Komponente dargestellt werden muss, wenn ein bestimmter Pfad in der Anwendung besucht wird.
%
\subsection{Grundfunktionen des Frameworks}
\label{subsec:grundfunktionen}
%
Module und Komponenten in Angular haben wir nun schon etwas ausführlicher betrachtet. Nun wollen wir uns weitere Funktionen und Bestandteile von Angular anschauen\footnote{Auf die weiteren Bestandteile von Angular wird hier nicht im Details eingegangen. Genauere Dokumentationen dazu sind online unter angular.io zu finden.}. Wie die einzelnen Bausteine umgesetzt und implementiert werden können sehen wir noch in Kapitel 4 Methodik am Fallbeispiel des 3D Konfigurators für Mehrwegbecher.
%
\paragraph{Bindungen}
%
\textit{\glqq Bindungen sind im Wesentlichen die Brücke zwischen der Darstellungsschicht und der Logikschicht\grqq } \cite{unlu_angular_2018}. Man kann beispielsweise Variablen oder Eigenschaften aus der TypeScript-Klasse in der Vorlage binden. Beispiel: Ich gebe die Variable \texttt{title} in der Vorlage als\texttt{ <h1> \{\{title\}\} </h1>} aus. Natürlich geht das ganze noch deutlich komplexer. Dadurch wird die Komponente mit ihrem Content dynamisch.
%
\paragraph{Direktiven}
%
Direktiven spielen innerhalb der Angular-Welt eine ähnlich wichtige Rolle wie Komponenten. Unter der Haube sieht man sogar, dass die voneinander erben. Im Wesentlichen ist es so, dass Direktiven in Vorlagen genutzt werden können. Und wie nutze ich sie? Indem ich Direktiven als Attribute auszeichne. Das heißt, Direktiven sind oft Attribute, die in ein HTML-Element hinzugefügt werden oder auch beispielsweise dadurch deklariert, dass an ein bestimmtes HTML-Element ein bestimmtes Attribut angehängt sein muss. Attribut-Direktiven machen eigentlich nichts anderes als eine Anpassung des Aussehens beziehungsweise eine Anpassung des Verhaltens eines Elements, das heißt, es manipuliert ein vorhandenes Element im Wesentlichen.\\
Analog dazu kann ich eine strukturelle Direktive in Form von ngFor benutzen. ngFor ist eine strukturelle Direktive, die etwas wahnsinnig Cleveres macht. Sie benutzt nämlich das Element, auf das sie angewendet wurde, als Vorlage, iteriert durch eine Liste, zum Beispiel eine unordered List, und packt diese Vorlage so oft in den DOM hinein, wie es Elemente in der Liste gibt.
%
\paragraph{Pipes}
%
Angular bietet uns die Möglichkeit zur Nutzung von Pipes. Sie dienen dazu, eine Ausgabe zu manipulieren. Überwiegend werden Pipes in Vorlagen genutzt.\\
Die Pipe manipuliert die Ausgabe und sorgt dafür, dass beispielsweise der Name in Großbuchstaben dargestellt wird. Analog lassen sich auch Pipes in Kette schalten. Das heißt, wenn ich eine Ausgabe habe, kann ich die wiederum in andere Pipes reinpacken.
%
\paragraph{Services}
%
Der Begriff Services in der Entwicklung ist so breit gefächert. Jede Programmiersprache versteht unter Services in gewissen Maßen was anderes. In der Angular-Welt sind Services Logiken, die allerdings View-unabhängig sind. Eine Komponente besteht aus Logik- und Darstellungsschicht, und in der Logikschicht habe ich ganz explizit die Logik, die nur für diese eine Komponente da ist.\\
Ein Service selber hat auch Logiken, die aber nicht allein für eine Komponente darstellen, sondern halt auch im Kontext in anderen Szenarien genutzt werden können. Super einfaches Beispiel dafür: Client-Server-Kommunikation. Wir können also einen Service erstellen, der das gesamte Handling des Logins steuert. Dieser Service verarbeitet dann via Passwort und Username die Client-Server-Kommunikation für das Login und empfängt vom Server dann das User-Objekt. Das kann aber an der anderen Stelle sein, dass ich zum Beispiel die Authentifizierung überprüfen möchte oder einfach nur das Geburtsjahr eines Users brauche, dann würde ich den gleichen Service in der anderen Komponente wieder nutzen und könnte dann entsprechend entweder auf die Eigenschaften des Services zurückgreifen, die es zuvor schon geholt hat, beziehungsweise auf Eigenschaftswerte, die es zuvor schon geholt hat, oder kann neue Requests triggern, Logger beispielsweise.
%
\subsection{Angular CLI}
Angular CLI ist ein mächtiges Tool. Es wird über die Command Line gesteuert. Sie kann beispielsweise verwendet werden, um ein neues Projekt anzulegen. Die CLI legt dann im Hintergrund die benötigte Projektstruktur mit allen benötigten Dateien in dem gewünschten Verzeichnis ab. Anschließend ist die Anwendung sogar schon lauffähig. Man kann sie ganz bequem über die CLI starten indem man den Developing Server verwendet. Mit Webpack wird das ganze Projekt gebündelt. Eine sehr große Hilfe können die Code Generatoren sein. Sie erzeugen uns automatisch Komponenten, Module oder Services. Dabei werden automatisch alle \texttt{@import-}Anweisungen vorgenommen. Wie das ganze konkret aussieht schauen wir uns bei der Umsetzung des Konfigurator noch einmal an. Angular bietet eine sehr gut verwendbare Testumgebung. Mit verschiedenen Modulen kann die Anwendung bis ins kleinste Detail getestet werden. Ganze User-Interaktionen können simuliert werden, auch auf verschiedenen Geräte-Klassifikationen. Wenn man die Anwendung veröffentlichen will, bietet das Framework über seine CLI eine build Funktion. Diese ermöglicht eine einfache und kompakte Lösung.
%
\subsection{Versionen}
Das Framework Angular setzt auf das System der semantischen Versionierung (\textit{SEMVER}). Die erste Version des Frameworks war \textit{AngularJS}. Schon die erste Version hatte das Ziel, ein strukturiertes und übersichtliches Framework zu sein. Mit der \textit{Version 2} wechselte die Programmiersprache von JavaScript zu TypeScript, welches von Microsoft entwickelt wurde. Das Framework wurde mit der Version 2 also komplett neu entwickelt. Es setzt aber großteils auf das alte Konzept von \textit{AngularJS}. Da das Router-Modul schon die Version 3 hatte, wurde die nächste Version von Angular komplett auf \textit{Version 4} angehoben, damit nun wieder alle Module auf der selben Version sind \cite{bohm_robin_angular_2017}. Mittlerweile ist die aktuelle Version des Framework \textit{Angular 7\footnote{Stand März 2019}}. Die aktuelle Version bringt ein paar Bugfixes mit sich sowie mehr Flexibilität. So können beispielsweise über die \textit{Angular CLI} ganz bequem alle Pakete automatisch aktualisiert werden. Anschließend sollte das Projekt mit der neuen Angular Version lauffähig sein (siehe \cite{steyer_ruhe_2018}).
%
%
%%%%%%%%%%%%%%%%%%%%%%%%%%%%%%%%%%%%%%%%%%%%%%%%%%%
%
% A U F B A U 
%
%%%%%%%%%%%%%%%%%%%%%%%%%%%%%%%%%%%%%%%%%%%%%%%%%%%
%
\section{Three.js}
\label{sec:three.js}
%
\textit{Die Entwicklung verbesserter 3D-Grafiken in webbasierten Anwendungen hat in letzter Zeit einen Schritt vorwärts getan, als Programmierer WebGL in die Mozilla Firefox Nightly-Builds und in WebKit integriert haben, das in Google Chrome und dem Safari-Browser von Apple verwendet wird. WebGL ist eine der am meisten entwickelten Bibliotheken, die von HTML5 unterstützt werden. }\cite{bahor_html5/webgl_2013}.\\
%
%
%%%%%%%%%%%%%%%%%%%%%%%%%%%%%%%%%%%%%%%%%%%%%%%%%%%
%
% A U F B A U 
%
%%%%%%%%%%%%%%%%%%%%%%%%%%%%%%%%%%%%%%%%%%%%%%%%%%%
%
\section{Responsive Webdesign}
\label{sec:responsive}
%
Der erste Eindruck ist wichtig. Ein Besucher benötigt nur 50 Millisekunden, um sich eine Meinung über eine Website zu bilden. Wenn also die Gestaltung der Webseite keinen guten ersten Eindruck hinterlässt, werden viele potenzielle Kunden einfach gehen\cite{webalive}. Laut einer Umfrage von \textit{clutch.co} sollen in 2019 nahezu alle Webseiten kleinerer Unternehmen mobil freundlich sein. Es ist offensichtlich, dass immer mehr mobile Geräte verwendet werden. Deshalb liegt es auch auf der Hand Webseiten für diese Geräteklasse anzupassen.
Laut einer Studie bevorzugen etwa 3/4 der Benutzer mobil freundliche Webseiten und würden sie auch wieder besuchen \cite{searchenginewatch}.\\
In seinem Artikel Responsive Web Design definiert Ethan Marcotte den Begriff mit 3 Säulen:\textit{ Flexible Layout-Grids, Flexible Bilder und Media Queries}. Er betont aber auch, dass es ein anderes Denken erfordert. \textit{Es ist ein mechanisches Konzept, das aus einer Person entsteht und auf endlichen, spezifischen Elementen basiert. Es gibt also keine klare Definition von Responsive Webdesign.} Oder anders gesagt: \textit{Ich kann Ihnen sagen, wie man Responsive Web Design macht. Wie wir die Dinge „reaktionsfähig“ machen, liegt an uns. Wir alle} \cite{responsive}.
%
%
%%%%%%%%%%%%%%%%%%%%%%%%%%%%%%%%%%%%%%%%%%%%%%%%%%%
%
% A U F B A U 
%
%%%%%%%%%%%%%%%%%%%%%%%%%%%%%%%%%%%%%%%%%%%%%%%%%%%
%
\section{Usability}
\label{sec:usability}
%
Usability ist ein umfassenderes Konzept, als allgemein unter „Benutzerfreundlichkeit“ oder „Benutzerfreundlichkeit“ verstanden wird. \\
Umfang, in dem ein System, ein Produkt oder eine Dienstleistung von bestimmten Benutzern verwendet werden kann, um bestimmte Ziele mit Effizienz, Effizienz und Zufriedenheit in einem bestimmten Nutzungskontext zu erreichen
Anmerkung 1 zum Begriff : Die „spezifizierten“ Benutzer, Ziele und Nutzungskontexte beziehen sich auf die bestimmte Kombination von Benutzern, Ziele und Nutzungskontext, für die die Verwendbarkeit in Betracht gezogen wird.
Anmerkung 2 zum Begriff : Das Wort „Usability“ wird auch als Qualifikationsmerkmal verwendet, um auf Designwissen, Kompetenzen, Aktivitäten und Designattribute zu verweisen, die zur Usability beitragen, wie beispielsweise Usability-Kenntnisse, Usability Professional, Usability Engineering, Usability-Methode, Usability-Bewertung Usability-Heuristik.
[QUELLE: ISO 9241-210: 2010, 2.13, geändert - Anmerkungen 1 und 2 wurden hinzugefügt.]
%
%
%%%%%%%%%%%%%%%%%%%%%%%%%%%%%%%%%%%%%%%%%%%%%%%%%%%
%
% A U F B A U 
%
%%%%%%%%%%%%%%%%%%%%%%%%%%%%%%%%%%%%%%%%%%%%%%%%%%%
%
\section{Performance}
\label{sec:performance}
%
\textit{Die Leistung vieler Websites hängt von der Auslastung der Website zu Spitzenzeiten unter verschiedenen Bedingungen ab. Leistungstests werden normalerweise in einer vernünftig simulierten Umgebung mit Hilfe von Leistungstestwerkzeugen durchgeführt. Die Leistung einer Website hängt jedoch von verschiedenen Parametern ab, und jeder Parameter muss unter verschiedenen Belastungsniveaus getestet werden.} Aufgrund der Komplexität von Websites ist es nicht möglich, einen gemeinsamen Nenner für Leistungsparameter zum Testen der Website zu zeichnen. Verschiedene Teile der Website müssen mit unterschiedlichen Parametern unter verschiedenen Bedingungen und Belastungsniveaus getestet werden. In solchen Fällen muss die Website in viele Komponenten zerlegt werden, die das Verhalten verschiedener Geschäftskomponenten darstellen. Diese Geschäftskomponenten werden verschiedenen Objekten zugeordnet, die das Verhalten und die Struktur des Teils der Website wirklich darstellen. Diese Objekte werden Leistungstests mit verschiedenen Parametern und Belastungsniveaus unterzogen. In diesem Dokument wird der neue Testprozess angesprochen, bei dem das Konzept der Zerlegung des Verhaltens der Website in testbare Komponenten verwendet wird, die auf testbare Objekte abgebildet werden. Diese überprüfbaren Objekte werden Leistungstests unter verschiedenen Leistungsparametern und Belastungsniveaus unterzogen.

		%---- Kapitel Methodik ----
		%
%%%%%%%%%%%%%%%%%%%%%%%%%%%%%%%%%%%%%%%%%%%%%%%%%%%%%%%%%%
%
%  A N F O R D E R U N G E N   &   K O N Z E P T Z I O N
%
%%%%%%%%%%%%%%%%%%%%%%%%%%%%%%%%%%%%%%%%%%%%%%%%%%%%%%%%%%
\chapter{Anforderungen und Konzeption}
\label{cha:methodik}
%
Unter funktionalen Anforderungen versteht man konkrekte Funktionalitäten, welche die Anwendung leisten soll. Die Anwendung lässt sich in mehrere Module aufteilen, die dem User zu Verfügung stehen. Im Folgenden werden die funktionalen Anforderungen erläutert.
%
%%%%%%%%%%%%%%%%%%%%%%%%%%%%%%%%%%%%%%%%%%%%%%%%%%%
%
% Problemanalyse
%
%%%%%%%%%%%%%%%%%%%%%%%%%%%%%%%%%%%%%%%%%%%%%%%%%%%
%
\section{Funktionale Anforderungen}
\label{sec:problemanalyse}
%
Unter funktionalen Anforderungen versteht man konkrekte Funktionalitäten, welche die Anwendung leisten soll. Die Anwendung lässt sich in mehrere Module aufteilen, die dem User zu Verfügung stehen. In diesem Kaptitel werden sowohl die funktionalen also auch die nicht funktionalen Anforderungen beschrieben. Es handelt sich zwar um ein echtes Projekt, wobei es sich vorerst um eine prototypische Entwicklung handelt. Als Anforderung vom Hersteller gilt nur, dass der Konfigurator einen Bestellvorgang für den Kunden und das Unternehmen vereinfacht. Dabei soll durch innovative Gestaltung und Programmierung des Konfigurators eine performante Lösung entwickelt werden. Das Ergebnis sollte zum bisherigen Webauftritt des Unternehmens passen. Nun wollen wir uns die Anforderungen im Details anschauen. Anschließend wird das Konzept des Webanwendung vorgestellt.

\renewcommand{\arraystretch}{1.8}
\begin{table}[h]
	\begin{tabular}{|r|c|p{8.5cm}|}
		\hline
		\textbf{Nr.} & \textbf{Bezeichnung} & \textbf{Beschreibung} \\
		\hline
		1 & Auswahl des Bechers & Der Benutzer kann zwischen verschiedenen Bechergrößen wählen. \\
		\hline
		2 & Interaktionen & Der Benutzer kann beispielsweise den Becher transformieren oder an den Becher heranzoomen. \\
		\hline
		3 & Upload des Designs & Der Benutzer kann nach sein Design entsprechende der Druckvorgaben hochlade.n \\
		\hline
		4 & Zusatzoptionen & Der Benutzer kann bestimmte Zusatzoptionen auswählen. \\
		\hline
		5 & Übersicht anzeigen & Der Benutzer kann sich eine Übersicht seiner Konfiguration ansehen. \\
		\hline
		6 & Konfiguration speichern & Der Benutzer kann seinen konfigurierten Becher speichern. \\
		\hline
	\end{tabular}
\caption{Tabellenüberschrift}
\end{table}
%
\paragraph{3D Vorschau des Bechers}
Zunächst ist wichtig, das die 3D Darstellung des Bechers umgesetzt wird. Dabei spielt das Design, welches der User hochladen wird, erst einmal keine Rolle. Schon beim Start wird Standardmäßig des 3D-Modell eines Bechers angezeigt. Von den Mehrwegbechern sind vier verschiedene Größen vorhanden. Dem Benutzer soll es möglich sein über das Konfigurationsmenü eine beliebige Größe auszuwählen. Ohne lange Wartezeiten muss sich das 3D-Modell des Bechers dementsprechend aktualisieren.\\
Falls der User schon ein Design hochgeladen hat und die Größe nun ändern möchte, wird das Design gelöscht und muss erneut hochgeladen werden. Grund dafür sind die unterschiedlichen Designvorgaben für jede Größe. Man soll vor dem Löschvorgang allerdings einen Warnhinweis bekommen, um ungewolltes Löschen zu vermeiden.
%
\paragraph{Interaktionen}
Da es sich um eine 3D Ansicht handelt, sollte man auch interagieren können. Der Becher kann von Benutzer 360 Grad auf der X-Achse und auf der Y-Achse 180 Grad gedreht werden. Auch eine Zoom-Funktion soll implementiert werden. Damit kann der User näher an den Becher heranzoomen. Es sollte jedoch ein maximaler Zoom eingestellt werden, damit die Ansicht realistisch bleibt. Außerdem soll zur einfachereren Bedienung ein Button erstellt werden, der den Becher wieder in seine Ursprungsposition bringt. Diese Interaktionen sollten sowohl mit der Maus als auch mit Touch-Eingabe funktionieren.
%
\paragraph{Upload des Designs}
Normalerweise kann ein Kunde einen Mehrwegbecher über Flyeralarm\footnote{https://www.flyeralarm.com/de/} drucken lassen. Dabei muss man sich an die Design-Vorgaben halten, die im Datenblatt\footnote{siehe Anhang} definiert sind. Diese unterscheiden sich je nach Größe des Bechers, da es sich ja um ein vollflächigen Druck handelt. Um den Benutzer Arbeit zu ersparen, sollte er das Design genau nach den Vorgaben der Druckerei hochladen. Beim Upload wird die Datei auf das korrekte Seitenverhältnis geprüft. Ohne lange Wartezeit wird nun das Design auf den Becher gerendert und der Kunde kann sich das Ergebnis anschauen. Es sollte auch möglich sein das Design mit einem anderen zu ersetzten. Wie schon erwähnt wird das Design bei ändern der Bechergröße entfernt. Als Dateiformat sind ausschließlich \texttt{png}-Dateien zulässig, weil damit Transarenzen dargestellt werden können.
%
\paragraph{Zusatzoptionen wählen}
Hierbei handelt es sich um optionale Zusatzfunktionen die nicht unbedingt im Rahmen des Bachlorprojektes umgesetzt werden. Eine Zusatzoption wäre das ein und ausblenden des Eichstrichs, der bei den Mehrwegbechern mit aufgedruckt wird. Eine weitere Zusatzangabe des Users wäre eine Eingabe der Stückzahl, die bestellt werden soll. Das wäre sinnvoll wenn der Konfigurator in Zukunft in den Bestellvorgang eingabaut werden soll. Andernfalls hätte der Kunde zumindest die Möglichkeit eine Preisvorschau in der Übersicht zu sehen.
%
\paragraph{Übersicht anzeigen}
Wie gerade erwähnt soll es eine Übersicht geben. In Form einer Tabelle soll dem Kunden nun noch einmal seine Auswahl zusammengefasst werden. Er kann dort die ausgewählte Bechergröße, das hochgeladene Design (den Dateinamen), eventuell die Stückzahl, Preisvorschau und so weiter, sehen. Optional kann eine Druck-Funktion implementiert werden, die es dem Benutzer ermöglicht die Übersicht inkulsive einer Abbildung des Bechers mit dem Design zu drucken bzw. in einer \texttt{pdf}-Datei zu speichern.
%
\paragraph{Konfiguration speichern}
Hierbei handelt es sich wieder um eine optionale Funktion, die wahrscheinlich nicht mehr im Rahmen der Bachlorarbeit umgesetzt werden kann. Wenn ein Benutzer einen Becher konfiguriert hat, ist es durchaus denkbar das er zu späteren Zeiten nocheinmal darauf zurückgreifen möchte. Oder er möchte verschiedene Versionen eines Bechers vergleichen. Deshalb wäre es sinnvoll eine Speicher-Funktion zu implementieren, die einfach und schnell den konfigurierten Becher wiederherstellen kann.
%
%%%%%%%%%%%%%%%%%%%%%%%%%%%%%%%%%%%%%%%%%%%%%%%%%%%
%
% Problemanalyse
%
%%%%%%%%%%%%%%%%%%%%%%%%%%%%%%%%%%%%%%%%%%%%%%%%%%%
%
\section{Nicht Funktionale Anforderungen}
\label{sec:problemanalyse}
%
Mit nicht funktionalen Anforderungen sind Funktionalitäten gemeint, welche die Anwendung \textbf{im Hintergrund} leisten soll. Diese sollen die Usablitiy und performance des Konfigurators verbessern und dem Benutzer somit ein innovatives Erlebnis bieten. In der Tabelle sehen sie eine Übersicht der nicht funktionalen Anforderungen.

\renewcommand{\arraystretch}{1.8}
\begin{table}
	\begin{tabular}{|r|c|p{8.5cm}|}
		\hline
		\textbf{Nr.} & \textbf{Bezeichnung} & \textbf{Beschreibung} \\
		\hline
		1 & Uploadfilter & Der Benutzer kann zwischen verschiedenen Bechergrößen wählen. \\
		\hline
		2 & Bedienungshilfen & Der Benutzer kann beispielsweise den Becher transformieren oder an den Becher heranzoomen. \\
		\hline
		3 & Realistische Design-Darstellung & Der Benutzer kann nach sein Design entsprechende der Druckvorgaben hochlade.n \\
		\hline
		4 & Responsive Webdesign & Der Benutzer kann bestimmte Zusatzoptionen auswählen. \\
		\hline
		5 & Benutzerfreundlich & Der Benutzer kann sich eine Übersicht seiner Konfiguration ansehen. \\
		\hline
		6 & Leistungstark & Der Benutzer kann seinen konfigurierten Becher speichern. \\
		\hline
	\end{tabular}
	\caption{Tabellenüberschrift}
\end{table}
%
\paragraph{Uploadfilter}
Das Design sollte realistisch gerendert werden. Dazu muss das Design wie in den Druckvorgaben hochgeladen werden. Sonst könnte es zu verzehrt aus sehen oder einfach unscharf. Nach dem Upload bekommt der User eine Rückmeldung, ob die Auflösung des Bildes in Ordnung ist. Falls nicht wird das Bild nicht gespeichert bzw. auf den Becher gerendert und der Benutzer soll das Design in der richtigen Auflösung hochladen. Ein ähnliches Vorgehen wird auch bei der Druckerei Fyleralarm bei einem Upload angewendet. Jedoch wird bei dem Konfigurator nur das Seitenverhältnis bzw. die Auflösung geprüft, nicht der Farbmodus oder andere Vorgaben zum Drucken.
%
\paragraph{Bedienungshilfen}
Ähnlich wie beim Upload des Designs soll der Kunde immer wieder Hinweise bekommen was zu tun ist. Beispielweise wie er den Becher drehen kann oder wo er die Größe des Bechers auswählen kann oder wo er die Übersicht seiner Auswahl sehen kann.
%
\paragraph{Realistisches Design-Rendering}
kubisches Problem
Auch was die 3D-Darstellung angeht, sind nicht viele Vorgaben gegeben, was einen weiten Spielraum lässt. Zunächst muss der Becher in die Szene geladen werden können. Anschließend besteht die Herausforderung darin, ein vom Benutzer hochgeladenes Design realistisch auf den Becher zu projizieren. Dabei muss die kubische Form des Bechers und die Upload-Vorgaben der Druckerei berücksichtigt werden. Der Upload-Vorgang muss einfach und verständlich sein, damit der Benutzer weiß wie er sein Design richtig auf den Becher bekommt. Dabei dürfen die Ladezeiten nicht zu lange sein. Die Anwendung muss möglichst performant laufen, unabhängig von der Plattform.
%
\paragraph{Responsive Webdesign}
Es ist wichtig das der Konfigurator auf verschiedenen Displaygrößen verwendet werden kann. Bei mobilen Geräten wird es schwierig die 3D Ansicht und das gesamte Menu zusammen anzuzeigen. Damit die Inhalte groß genug und gut lesbar sein, wird das Menu auf mobilen Gräten also \textit{Off-Canvas-Menu} implementiert. Somit wird zum einen der Becher schön groß dargestellt und zum anderen werden Menu-Items groß und gut lesbar sein. Der Benutzer eines Smartphones sollte nach Möglichkeit den gleichen Funktionsumfang genießen wie ein Benutzer auf Desktop Geräten. Dabei sollte möglichst innovativ gestalet werden.
%
\paragraph{Benutzerfreundlichkeit}
Die Webanwendung soll benutzerfreundlich sein. Dazu tragen die schon erwähnten Bedienungshilfen bei. Wichtig ist, das der Benutzer immer weiß was der nächste Schritt ist. Um dies zu erreichen ist eine einfache und übersichtliche Gestaltung ist nötig. Sie sollte auf das wesentliche beschränkt sein und nicht überladen wirken.
%
\paragraph{Leistungsstark}
Wie die Studie im Kapitel Performance gezeigt hat, mögen Benutzer keine langen Wartezeiten. Bei dem 3D-Konfigurator soll das Laden der 3D Modelle nicht länger als 2ms dauern. Das rendern des Designs auf den Becher ebenfalls nicht länger als 2ms. Die Länge des Uploads vom Design hängt von der Internetverbindung ab, sollte allerdings auch schnell abgeschlossen sein. Das Transfomieren des Models in der 3D szene sollte flüssig funktionieren und keine Verzögerungen enthalten. Alles in allem sollte die gesamte Anwendung performant laufen und im Bezug auf die Leistung an eine Desktopanwendung erinnern. Besonders interessant wird dieser Aspket auf mobilen Geräten, denn auch dort sollte die Anwendung möglichst stabil laufen.
%
%
%%%%%%%%%%%%%%%%%%%%%%%%%%%%%%%%%%%%%%%%%%%%%%%%%%%
%
% Konzept
%
%%%%%%%%%%%%%%%%%%%%%%%%%%%%%%%%%%%%%%%%%%%%%%%%%%%
%
\section{Konzeption}
\label{sec:konzept}
%
Bevor es an die Programmierung geht, sollte man sich im klaren sein, wie die Anwendung im Groben aufgebaut sein wird. Nachdem alle Funktionen und Vorgaben bekannt sind, kann man ein sogenanntes UML (Unified Modeling Language) entwerfen. Hierbei handelt es sich um eine einfache Modellierungssprache. Hierbei werden benötigte Klassen, Komponenten und Module definiert. Das hilft beim Programmieren den Überblick zu behalten. Im Folgenden  ist das UML für den Konfigurator zu sehen.
• Konzept (UML und weitere Diagramme)
• Mokups der Einzelnen Screens
• Responsive Layout Ideen \\
%
%
%%%%%%%%%%%%%%%%%%%%%%%%%%%%%%%%%%%%%%%%%%%%%%%%%%%
%
% Umsetzung & Implementierung
%
%%%%%%%%%%%%%%%%%%%%%%%%%%%%%%%%%%%%%%%%%%%%%%%%%%%
%
\section{Umsetzung \& Imlementierung}
\label{sec:umsetzung}
%
Diese Vorlage verwendet die \textit{memoir}-LaTeX-Klasse von Peter Wilson und erzeugt somit ein Buchdokument, das auch zweiseitig gedruckt werden sollte, da zwischen den Kapiteln Leerseiten erzeugt werden. Einstiegspunkt für die Bachelorarbeit ist die Datei \texttt{abschlussarbeit.tex}. Tragen Sie zunächst den Titel der Arbeit und alle Namen ein. Bei hochschulinternen Bachlorarbeiten fallen die Eintragungen \textit{durchgeführt am} und \textit{Betreuer in der Firma} natürlich weg. 
\paragraph{3D Zylinder für Design rendering}
%https://threejs.org/docs/#api/en/geometries/CylinderBufferGeometry\\
%https://threejs.org/docs/#api/en/core/BufferGeometry\\
%https://codepen.io/JasonStoltz/pen/NaBOGm (transparente Designs) \\
%https://www.flyeralarm.com/sheets/de/becher_kstoff_200ml.pdf\\
%https://stackoverflow.com/questions/55283187/threejs-texture-fit-uv-map\\

	%---- Beginn Anhang ----
	\appendix	% resets numbering to alphabetic style
	\include{anhang/anhang}
	%---- Ende Anhang ----

	\cleardoublepage



	% ---- the backmatter ----
	\backmatter
	
		%---- include Glossar ----
		%
%%%%%%%%%%%%%%%%%%%%%%%%%%%%%%%%%%%%%%%%%%%%%%%%%%%
%
% G L O S S A R
%
%%%%%%%%%%%%%%%%%%%%%%%%%%%%%%%%%%%%%%%%%%%%%%%%%%%
\chapter{Glossar}
\label{cha:glossar}


\begin{tabular}{p{3cm} p{12cm}}

\textbf{CLI} & Command Line Interface\\
\\
\textbf{SEMVER} & Semantische Versionierung \\
\\
\textbf{UML} & Unified Modeling Language \\
\\
\end{tabular}


		%---- Literaturverzeichnis ----
%		\begin{footnotesize}
			\bibliography{abschlussarbeit}
			\bibliographystyle{alphadin}
%		\end{footnotesize}
		
		%---- Own publications again ----
		\cleardoublepage

		%---- Indexverzeichnis ----	
		\cleardoublepage
%		\pagestyle{index}
%		\renewcommand{\chaptermark}[1]{}
		\renewcommand{\preindexhook}{ % Die erste angegebene Seitenzahl ist gewöhnlich, aber nicht immer, die erste Referenz zum entsprechenden Eintrag.
		\vskip\onelineskip}
		\indexintoc
		\printindex
		\cleardoublepage

	
\end{document}
